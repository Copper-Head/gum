% Generated by Sphinx.
\def\sphinxdocclass{report}
\documentclass[letterpaper,10pt,english]{sphinxmanual}
\usepackage[utf8]{inputenc}
\DeclareUnicodeCharacter{00A0}{\nobreakspace}
\usepackage{cmap}
\usepackage[T1]{fontenc}
\usepackage{babel}
\usepackage{times}
\usepackage[Bjarne]{fncychap}
\usepackage{longtable}
\usepackage{sphinx}
\usepackage{multirow}


\title{gum Documentation}
\date{June 16, 2014}
\release{0}
\author{Ilia Kurenkov}
\newcommand{\sphinxlogo}{}
\renewcommand{\releasename}{Release}
\makeindex

\makeatletter
\def\PYG@reset{\let\PYG@it=\relax \let\PYG@bf=\relax%
    \let\PYG@ul=\relax \let\PYG@tc=\relax%
    \let\PYG@bc=\relax \let\PYG@ff=\relax}
\def\PYG@tok#1{\csname PYG@tok@#1\endcsname}
\def\PYG@toks#1+{\ifx\relax#1\empty\else%
    \PYG@tok{#1}\expandafter\PYG@toks\fi}
\def\PYG@do#1{\PYG@bc{\PYG@tc{\PYG@ul{%
    \PYG@it{\PYG@bf{\PYG@ff{#1}}}}}}}
\def\PYG#1#2{\PYG@reset\PYG@toks#1+\relax+\PYG@do{#2}}

\expandafter\def\csname PYG@tok@gd\endcsname{\def\PYG@tc##1{\textcolor[rgb]{0.63,0.00,0.00}{##1}}}
\expandafter\def\csname PYG@tok@gu\endcsname{\let\PYG@bf=\textbf\def\PYG@tc##1{\textcolor[rgb]{0.50,0.00,0.50}{##1}}}
\expandafter\def\csname PYG@tok@gt\endcsname{\def\PYG@tc##1{\textcolor[rgb]{0.00,0.27,0.87}{##1}}}
\expandafter\def\csname PYG@tok@gs\endcsname{\let\PYG@bf=\textbf}
\expandafter\def\csname PYG@tok@gr\endcsname{\def\PYG@tc##1{\textcolor[rgb]{1.00,0.00,0.00}{##1}}}
\expandafter\def\csname PYG@tok@cm\endcsname{\let\PYG@it=\textit\def\PYG@tc##1{\textcolor[rgb]{0.25,0.50,0.56}{##1}}}
\expandafter\def\csname PYG@tok@vg\endcsname{\def\PYG@tc##1{\textcolor[rgb]{0.73,0.38,0.84}{##1}}}
\expandafter\def\csname PYG@tok@m\endcsname{\def\PYG@tc##1{\textcolor[rgb]{0.13,0.50,0.31}{##1}}}
\expandafter\def\csname PYG@tok@mh\endcsname{\def\PYG@tc##1{\textcolor[rgb]{0.13,0.50,0.31}{##1}}}
\expandafter\def\csname PYG@tok@cs\endcsname{\def\PYG@tc##1{\textcolor[rgb]{0.25,0.50,0.56}{##1}}\def\PYG@bc##1{\setlength{\fboxsep}{0pt}\colorbox[rgb]{1.00,0.94,0.94}{\strut ##1}}}
\expandafter\def\csname PYG@tok@ge\endcsname{\let\PYG@it=\textit}
\expandafter\def\csname PYG@tok@vc\endcsname{\def\PYG@tc##1{\textcolor[rgb]{0.73,0.38,0.84}{##1}}}
\expandafter\def\csname PYG@tok@il\endcsname{\def\PYG@tc##1{\textcolor[rgb]{0.13,0.50,0.31}{##1}}}
\expandafter\def\csname PYG@tok@go\endcsname{\def\PYG@tc##1{\textcolor[rgb]{0.20,0.20,0.20}{##1}}}
\expandafter\def\csname PYG@tok@cp\endcsname{\def\PYG@tc##1{\textcolor[rgb]{0.00,0.44,0.13}{##1}}}
\expandafter\def\csname PYG@tok@gi\endcsname{\def\PYG@tc##1{\textcolor[rgb]{0.00,0.63,0.00}{##1}}}
\expandafter\def\csname PYG@tok@gh\endcsname{\let\PYG@bf=\textbf\def\PYG@tc##1{\textcolor[rgb]{0.00,0.00,0.50}{##1}}}
\expandafter\def\csname PYG@tok@ni\endcsname{\let\PYG@bf=\textbf\def\PYG@tc##1{\textcolor[rgb]{0.84,0.33,0.22}{##1}}}
\expandafter\def\csname PYG@tok@nl\endcsname{\let\PYG@bf=\textbf\def\PYG@tc##1{\textcolor[rgb]{0.00,0.13,0.44}{##1}}}
\expandafter\def\csname PYG@tok@nn\endcsname{\let\PYG@bf=\textbf\def\PYG@tc##1{\textcolor[rgb]{0.05,0.52,0.71}{##1}}}
\expandafter\def\csname PYG@tok@no\endcsname{\def\PYG@tc##1{\textcolor[rgb]{0.38,0.68,0.84}{##1}}}
\expandafter\def\csname PYG@tok@na\endcsname{\def\PYG@tc##1{\textcolor[rgb]{0.25,0.44,0.63}{##1}}}
\expandafter\def\csname PYG@tok@nb\endcsname{\def\PYG@tc##1{\textcolor[rgb]{0.00,0.44,0.13}{##1}}}
\expandafter\def\csname PYG@tok@nc\endcsname{\let\PYG@bf=\textbf\def\PYG@tc##1{\textcolor[rgb]{0.05,0.52,0.71}{##1}}}
\expandafter\def\csname PYG@tok@nd\endcsname{\let\PYG@bf=\textbf\def\PYG@tc##1{\textcolor[rgb]{0.33,0.33,0.33}{##1}}}
\expandafter\def\csname PYG@tok@ne\endcsname{\def\PYG@tc##1{\textcolor[rgb]{0.00,0.44,0.13}{##1}}}
\expandafter\def\csname PYG@tok@nf\endcsname{\def\PYG@tc##1{\textcolor[rgb]{0.02,0.16,0.49}{##1}}}
\expandafter\def\csname PYG@tok@si\endcsname{\let\PYG@it=\textit\def\PYG@tc##1{\textcolor[rgb]{0.44,0.63,0.82}{##1}}}
\expandafter\def\csname PYG@tok@s2\endcsname{\def\PYG@tc##1{\textcolor[rgb]{0.25,0.44,0.63}{##1}}}
\expandafter\def\csname PYG@tok@vi\endcsname{\def\PYG@tc##1{\textcolor[rgb]{0.73,0.38,0.84}{##1}}}
\expandafter\def\csname PYG@tok@nt\endcsname{\let\PYG@bf=\textbf\def\PYG@tc##1{\textcolor[rgb]{0.02,0.16,0.45}{##1}}}
\expandafter\def\csname PYG@tok@nv\endcsname{\def\PYG@tc##1{\textcolor[rgb]{0.73,0.38,0.84}{##1}}}
\expandafter\def\csname PYG@tok@s1\endcsname{\def\PYG@tc##1{\textcolor[rgb]{0.25,0.44,0.63}{##1}}}
\expandafter\def\csname PYG@tok@gp\endcsname{\let\PYG@bf=\textbf\def\PYG@tc##1{\textcolor[rgb]{0.78,0.36,0.04}{##1}}}
\expandafter\def\csname PYG@tok@sh\endcsname{\def\PYG@tc##1{\textcolor[rgb]{0.25,0.44,0.63}{##1}}}
\expandafter\def\csname PYG@tok@ow\endcsname{\let\PYG@bf=\textbf\def\PYG@tc##1{\textcolor[rgb]{0.00,0.44,0.13}{##1}}}
\expandafter\def\csname PYG@tok@sx\endcsname{\def\PYG@tc##1{\textcolor[rgb]{0.78,0.36,0.04}{##1}}}
\expandafter\def\csname PYG@tok@bp\endcsname{\def\PYG@tc##1{\textcolor[rgb]{0.00,0.44,0.13}{##1}}}
\expandafter\def\csname PYG@tok@c1\endcsname{\let\PYG@it=\textit\def\PYG@tc##1{\textcolor[rgb]{0.25,0.50,0.56}{##1}}}
\expandafter\def\csname PYG@tok@kc\endcsname{\let\PYG@bf=\textbf\def\PYG@tc##1{\textcolor[rgb]{0.00,0.44,0.13}{##1}}}
\expandafter\def\csname PYG@tok@c\endcsname{\let\PYG@it=\textit\def\PYG@tc##1{\textcolor[rgb]{0.25,0.50,0.56}{##1}}}
\expandafter\def\csname PYG@tok@mf\endcsname{\def\PYG@tc##1{\textcolor[rgb]{0.13,0.50,0.31}{##1}}}
\expandafter\def\csname PYG@tok@err\endcsname{\def\PYG@bc##1{\setlength{\fboxsep}{0pt}\fcolorbox[rgb]{1.00,0.00,0.00}{1,1,1}{\strut ##1}}}
\expandafter\def\csname PYG@tok@kd\endcsname{\let\PYG@bf=\textbf\def\PYG@tc##1{\textcolor[rgb]{0.00,0.44,0.13}{##1}}}
\expandafter\def\csname PYG@tok@ss\endcsname{\def\PYG@tc##1{\textcolor[rgb]{0.32,0.47,0.09}{##1}}}
\expandafter\def\csname PYG@tok@sr\endcsname{\def\PYG@tc##1{\textcolor[rgb]{0.14,0.33,0.53}{##1}}}
\expandafter\def\csname PYG@tok@mo\endcsname{\def\PYG@tc##1{\textcolor[rgb]{0.13,0.50,0.31}{##1}}}
\expandafter\def\csname PYG@tok@mi\endcsname{\def\PYG@tc##1{\textcolor[rgb]{0.13,0.50,0.31}{##1}}}
\expandafter\def\csname PYG@tok@kn\endcsname{\let\PYG@bf=\textbf\def\PYG@tc##1{\textcolor[rgb]{0.00,0.44,0.13}{##1}}}
\expandafter\def\csname PYG@tok@o\endcsname{\def\PYG@tc##1{\textcolor[rgb]{0.40,0.40,0.40}{##1}}}
\expandafter\def\csname PYG@tok@kr\endcsname{\let\PYG@bf=\textbf\def\PYG@tc##1{\textcolor[rgb]{0.00,0.44,0.13}{##1}}}
\expandafter\def\csname PYG@tok@s\endcsname{\def\PYG@tc##1{\textcolor[rgb]{0.25,0.44,0.63}{##1}}}
\expandafter\def\csname PYG@tok@kp\endcsname{\def\PYG@tc##1{\textcolor[rgb]{0.00,0.44,0.13}{##1}}}
\expandafter\def\csname PYG@tok@w\endcsname{\def\PYG@tc##1{\textcolor[rgb]{0.73,0.73,0.73}{##1}}}
\expandafter\def\csname PYG@tok@kt\endcsname{\def\PYG@tc##1{\textcolor[rgb]{0.56,0.13,0.00}{##1}}}
\expandafter\def\csname PYG@tok@sc\endcsname{\def\PYG@tc##1{\textcolor[rgb]{0.25,0.44,0.63}{##1}}}
\expandafter\def\csname PYG@tok@sb\endcsname{\def\PYG@tc##1{\textcolor[rgb]{0.25,0.44,0.63}{##1}}}
\expandafter\def\csname PYG@tok@k\endcsname{\let\PYG@bf=\textbf\def\PYG@tc##1{\textcolor[rgb]{0.00,0.44,0.13}{##1}}}
\expandafter\def\csname PYG@tok@se\endcsname{\let\PYG@bf=\textbf\def\PYG@tc##1{\textcolor[rgb]{0.25,0.44,0.63}{##1}}}
\expandafter\def\csname PYG@tok@sd\endcsname{\let\PYG@it=\textit\def\PYG@tc##1{\textcolor[rgb]{0.25,0.44,0.63}{##1}}}

\def\PYGZbs{\char`\\}
\def\PYGZus{\char`\_}
\def\PYGZob{\char`\{}
\def\PYGZcb{\char`\}}
\def\PYGZca{\char`\^}
\def\PYGZam{\char`\&}
\def\PYGZlt{\char`\<}
\def\PYGZgt{\char`\>}
\def\PYGZsh{\char`\#}
\def\PYGZpc{\char`\%}
\def\PYGZdl{\char`\$}
\def\PYGZhy{\char`\-}
\def\PYGZsq{\char`\'}
\def\PYGZdq{\char`\"}
\def\PYGZti{\char`\~}
% for compatibility with earlier versions
\def\PYGZat{@}
\def\PYGZlb{[}
\def\PYGZrb{]}
\makeatother

\begin{document}

\maketitle
\tableofcontents
\phantomsection\label{index::doc}


Contents:


\chapter{Gum for Python 2.X}
\label{test:module-gum}\label{test:gum-for-python-2-x}\label{test:welcome-to-gum-s-documentation}\label{test::doc}\index{gum (module)}
test docstring
\index{add\_newlines() (in module gum)}

\begin{fulllineitems}
\phantomsection\label{test:gum.add_newlines}\pysiglinewithargsret{\code{gum.}\bfcode{add\_newlines}}{\emph{input\_iter}, \emph{newline='\textbackslash{}n'}, \emph{item\_type=\textless{}type `str'\textgreater{}}}{}
Takes a iterable, adds a new line character to the end of each of its
members and then returns a generator of the newly created items.
The idea is to convert some sequence that was created with no concern for
spliting it into lines into something that will produce a text file.
It is assumed that the only input types will be sequences of lists or
strings, because these are the only practically reasonable types to be
written to files.
It is also assumed that by default the sequence will consist of strings and
that the lines will be separated by a Unix newline character.
This behavior can be changed by passing different newline and/or itemType
arguments.

\end{fulllineitems}

\index{create\_debug\_log() (in module gum)}

\begin{fulllineitems}
\phantomsection\label{test:gum.create_debug_log}\pysiglinewithargsret{\code{gum.}\bfcode{create\_debug\_log}}{\emph{base='error'}, \emph{ext='.log'}, \emph{separator='\_'}, \emph{app='DEFAULT'}}{}
wrapper for creating a logger.
\begin{quote}\begin{description}
\item[{Parameters}] \leavevmode
\textbf{fileNameBase} -- base for the log file name to which date, time,

\end{description}\end{quote}

and the extension are later attached.
:type ext: string
:param ext: string for an extension
:type separator: string
:param separator: character used to separate different parts of the
filename
:type app: string
:param app: name for the application that generates the error

\end{fulllineitems}

\index{create\_row\_dicts() (in module gum)}

\begin{fulllineitems}
\phantomsection\label{test:gum.create_row_dicts}\pysiglinewithargsret{\code{gum.}\bfcode{create\_row\_dicts}}{\emph{fields}, \emph{data}, \emph{fill\_val='NA'}}{}
Helper generator function for the write\_to\_table(). Collecting data
is often much more efficient and clear when this data is stored in tuples
or lists, not dictionaries.
Python's csv DictWriter class requires that it be passed a sequence of 
dictionaries, however.
This function takes a header list of column names as well as some data in
the form of a sequence of rows (which can be tuples or lists) and converts
every row in the data to a dictionary usable by DictWriter.

\end{fulllineitems}

\index{find\_something() (in module gum)}

\begin{fulllineitems}
\phantomsection\label{test:gum.find_something}\pysiglinewithargsret{\code{gum.}\bfcode{find\_something}}{\emph{smthng}, \emph{string}, \emph{All=False}}{}
I'm not sure I should keep this

\end{fulllineitems}

\index{gen\_file\_paths() (in module gum)}

\begin{fulllineitems}
\phantomsection\label{test:gum.gen_file_paths}\pysiglinewithargsret{\code{gum.}\bfcode{gen\_file\_paths}}{\emph{dir\_name}, \emph{filter\_func=None}}{}
A function for wrapping all the os.path commands involved in listing files
in a directory, then turning file names into file paths by concatenating
them with the directory name.
This also optionally supports filtering file names using filter\_func.

\end{fulllineitems}

\index{pickle\_data() (in module gum)}

\begin{fulllineitems}
\phantomsection\label{test:gum.pickle_data}\pysiglinewithargsret{\code{gum.}\bfcode{pickle\_data}}{\emph{data}, \emph{file\_name}, \emph{ext='.picl'}}{}
wrapper for picling any data.
:type data: any
:param data: python object to be pickled
:type fileName: string
:param fileName: specifies the name of the pickled file
:type ext: string
:param ext: adds and extension to the file name

\end{fulllineitems}

\index{read\_table() (in module gum)}

\begin{fulllineitems}
\phantomsection\label{test:gum.read_table}\pysiglinewithargsret{\code{gum.}\bfcode{read\_table}}{\emph{file\_name}, \emph{function=None}, \emph{**fmtparams}}{}
Function that simplifies reading it table files of any kind.

\end{fulllineitems}

\index{subset\_dict() (in module gum)}

\begin{fulllineitems}
\phantomsection\label{test:gum.subset_dict}\pysiglinewithargsret{\code{gum.}\bfcode{subset\_dict}}{\emph{src\_dict}, \emph{relevants}, \emph{replace=False}, \emph{exclude=False}}{}
Given some keys and a dictionary returns a dictionary with only
specified keys. Assumes the keys are in fact present and will raise an
error if this is not the case

\end{fulllineitems}

\index{write\_to\_table() (in module gum)}

\begin{fulllineitems}
\phantomsection\label{test:gum.write_to_table}\pysiglinewithargsret{\code{gum.}\bfcode{write\_to\_table}}{\emph{file\_name}, \emph{data}, \emph{header=None}, \emph{**kwargs}}{}
Writes data to file specified by filename.
\begin{quote}\begin{description}
\item[{Parameters}] \leavevmode\begin{itemize}
\item {} 
\textbf{file\_name} (\emph{string}) -- name of the file to be created

\item {} 
\textbf{data} (\emph{iterable}) -- some iterable of dictionaries each of which

\end{itemize}

\end{description}\end{quote}

must not contain keys absent in the `header' argument
:type header: list
:param header: list of columns to appear in the output
:type {\color{red}\bfseries{}**}kwargs: dict
:param {\color{red}\bfseries{}**}kwargs: parameters to be passed to DictWriter.
For instance, restvals specifies what to set empty cells to by default or
`dialect' loads a whole host of parameters associated with a certain csv
dialect (eg. ``excel'').

\end{fulllineitems}

\index{write\_to\_txt() (in module gum)}

\begin{fulllineitems}
\phantomsection\label{test:gum.write_to_txt}\pysiglinewithargsret{\code{gum.}\bfcode{write\_to\_txt}}{\emph{file\_name}, \emph{data}, \emph{mode='w'}, \emph{AddNewLines=False}, \emph{**kwargs}}{}
Writes data to a text file.
\begin{quote}\begin{description}
\item[{Parameters}] \leavevmode\begin{itemize}
\item {} 
\textbf{fName} (\emph{string}) -- name of the file to be created

\item {} 
\textbf{data} (\emph{iterable}) -- some iterable of strings or lists of strings (not a string)

\item {} 
\textbf{addNewLines} (\emph{bool}) -- determines if it's necessary to add newline chars to

\end{itemize}

\end{description}\end{quote}

members of list
:type kwargs: dict
:param kwargs: key word args to be passed to list\_to\_plain\_text, if needed

\end{fulllineitems}



\chapter{Indices and tables}
\label{index:indices-and-tables}\begin{itemize}
\item {} 
\emph{genindex}

\item {} 
\emph{modindex}

\item {} 
\emph{search}

\end{itemize}


\renewcommand{\indexname}{Python Module Index}
\begin{theindex}
\def\bigletter#1{{\Large\sffamily#1}\nopagebreak\vspace{1mm}}
\bigletter{g}
\item {\texttt{gum}}, \pageref{test:module-gum}
\end{theindex}

\renewcommand{\indexname}{Index}
\printindex
\end{document}
